\todo[inline]{ProRail: km's geluidsmuur}

\section{Section}


    \subsection{Subsection}\label{subsec:subsection}

        \href{https://www.mjpgspoor.nl/}{ProRail, 72km
        geluidsmuur MJPG}

        \todo[inline]{RWS: km's geluidsmuur}
        \href{https://www.prorail.nl/programmas/mjpg}{ProRail MJPG}


        \todo[inline]{check alle missing refs}

        \todo[inline]{fruit walls}

        \href{https://www.nrc.nl/nieuws/2022/09/13/waarom-de-social-design-van-studio-drift-hypocriet-is-2-a4141699}
        {Waarom de social design van Studio Drift hypocriet is}

        \begin{displayquote}
            Toch is de hypocrisie van Drift het ergste.
            Terwijl de wereld in rampen en ontij is gehuld
            –-klimaatverandering, bosbranden-–
            gebruikt Studio Drift de ondergang van de wereld
            om zwevende lichtjes en gladgepolijste blokkendozen mee aan de man te brengen
        \end{displayquote}

        \href{https://www.nrc.nl/nieuws/2022/06/13/tachtig-suggesties-voor-een-betere-wereld-2-a4133270}
        {Tachtig suggesties voor een betere wereld}

        \begin{displayquote}
            De echte wereld en de gedroomde klimaatneutrale wereld lijken hier wel erg ver weg.
            Wat zijn de consequenties als we van vijgen plaatmateriaal maken en van schillen lampjes?
            Je kunt er als bezoeker slechts naar raden.
            In de expositie hangt een scherm waar de twee verantwoordelijke conservatoren
            in beeld komen en statements geven.
            De gepresenteerde ontwerpen schreeuwen juist om wetenschappelijke
            onderbouwing, om toelichtingen door ecologen en klimaatwetenschappers.
            Met een heldere en wetenschappelijk stevig onderbouwde catalogus voelt
            die tentoonstelling minder gemakzuchtig en een stuk urgenter aan.
            It’s Our F***ing Back-yard is een gemiste kans op een betekenisvol statement.
        \end{displayquote}

        Context \textbf{Rammed-earth}
        \begin{enumerate}
            \item AD digital landscape
            \item Sym[bio]scape AADRL
            \item Boltshauser
        \end{enumerate}

        bio-based stabilizer vs

        Context \textbf{Shot-earth}
        \begin{enumerate}
            \item Curto
            \item Petit freres
            \item Norman Hack
        \end{enumerate}

        Context \textbf{landscape robotics}
        \begin{enumerate}
            \item hurkxkens~\cite{hurkxkens_robotic_2020, hurkxkens_shifting_2021}
            \item Ryan Johnson's Phd~\cite{johns_autonomous_2020}
        \end{enumerate}

        Context \textbf{3DPSE}

        \begin{enumerate}
            \item IAAC
            \item WASP
            \item Drone, Mudd Architects
            \item St Gallen
            \item ETH Lehmag AG pavilion
            \item ETH Oosterijkse project, Venice biennale
        \end{enumerate}


        Check that references work out alright \cite{khan_sustainability_2021}

        \begin{enumerate}
            \item ~\cite{lim_3d_2020}
            \item ~\cite{curto_shot-earth_2020}
            \item ~\cite{crane_geodesics_2013}
        \end{enumerate}

        use of clever ref: \cref{subsec:subsection}

        fun predefined symbols: \kjoeb -- \cotwee -- \carboneq


